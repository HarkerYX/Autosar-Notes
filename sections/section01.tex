\section{Motivation}\index{Motivation}
The automotive industry is today the sixth largest economy in the world, produc-
ing around 70 million cars every year and making an important contribution to
government revenues all around the world. As for other industries, significant
improvements in functionalities, performance, comfort, safety, etc. are provided by
electronic and software technologies. Indeed, since 1990, the sector of embedded elec-
tronics, and more precisely embedded software, has been increasing at an annual rate
shared between electronic and software components. These general trends have led to
currently embedding up to 500 MB on more than 70 microprocessors connected
on communication networks. The following are some of the various examples. Figure
1.1 shows an electronic architecture embedded in a Laguna (source: Renault French
carmaker) illustrating several computers interconnected and controlling the engine,
the wipers, the lights, the doors, and the suspension or providing a support for inter-
action with the driver or the passengers. In 2004, the embedded electronic system of a
Volkswagen Phaeton was composed of more than 10,000 electrical devices, 61 micro-
processors, three controller area networks (CAN) that support the exchanges of 2500
pieces of data, several subnetworks, and one multimedia bus. In the Volvo S70,
two networks support the communication between the microprocessors controlling
the mirrors, those controlling the doors and those controlling the transmission system
and, for example, the position of the mirrors is automatically controlled according to
the sense the vehicle is going and the volume of the radio is adjusted to the vehi-
cle speed, information provided, among others, by the antilock braking system (ABS)
controller. In a recent Cadillac, when an accident causes an airbag to inflate, its micro-
controller emits a signal to the embedded global positioning system (GPS) receiver
that then communicates with the cell phone, making it possible to give the vehicle’s
position to the rescue service. The software code size of the Peugeot CX model (source:
PSA Peugeot Citroen French carmarker) was 1.1 KB in 1980, and 2 MB for the 607
model in 2000. These are just a few examples, but there are many more that could
illustrate this very large growth of embedded electronic systems in modern vehicles.
The automotive industry has evolved rapidly and will evolve even more rapidly
under the influence of several factors such as pressure from state legislation, pressure
from customers, and technological progress (hardware and software aspects). Indeed,
a great surge for the development of electronic control systems came through the
regulation concerning air pollution. But we must also consider the pressure from
Vehicle Functional Domains and Their Requirements
consumers for more performance (at lower fuel consumption), comfort, and safety.
Add to all this the fact that satisfying these needs and obligations is only possible
because of technological progress.
Electronic technology has made great strides and nowadays the quality of electronic
components—performance, robustness, and reliability—enables using them even for
critical systems. At the same time, the decreasing cost of electronic technology allows
them to be used to support any function in a car. Furthermore, in the last decade,
several automotive-embedded networks such as local interconnect networks (LIN),
CAN, TTP/C, FlexRay, MOST, and IDB-1394 were developed. This has led to the con-
cept of multiplexing, whose principal advantage is a significant reduction in the wiring
cost as well as the flexibility it gives to designers; data (e.g., vehicle speed) sampled by
one microcontroller becomes available to distant functions that need them with no
additional sensors or links.
Another technological reason for the increase of automotive embedded systems is
the fact that these new hardware and software technologies facilitate the introduction
of functions whose development would be costly or not even feasible if using only
mechanical or hydraulic technology. Consequently, they allow to satisfy the end user
requirements in terms of safety, comfort, and even costs. Well-known examples are
electronic engine control, ABS, electronic stability program (ESP), active suspension,
etc. In short, thanks to these technologies, customers can buy a safe, efficient, and
personalized vehicle, while carmakers are able to master the differentiation between
product variations and innovation (analysts have stated that more than 80\% of inno-
vation, and therefore of added value, will be obtained thanks to electronic systems).
Furthermore, it also has to be noted that some functions can only be achieved through
digital systems. The following are some examples: (1) the mastering of air pollution
can only be achieved by controlling the engine with complex control laws; (2) new
engine concepts could not be implemented without an electronic control; (3) mod-
ern stability control systems (e.g., ESP), which are based on close interaction between
the engine, steering, and braking controllers, can be efficiently implemented using an
embedded network.
Last, multimedia and telematic applications in cars are increasing rapidly due to
consumer pressure; a vehicle currently includes electronic equipment like hand-free
phones, audio/radio devices, and navigation systems. For the passengers, a lot of
entertainment devices, such as video equipment and communication with the out-
side world are also available. These kinds of applications have little to do with the
vehicle’s operation itself; nevertheless they increase significantly as part of the software
included in a car.
In short, it seems that electronic systems enable limitless progress. But are elec-
tronics free from any outside pressure? No. Unfortunately, the greatest pressure on
electronics is cost!
Keeping in mind that the primary function of a car is to provide a safe and efficient
means of transport, we can observe that this continuously evolving “electronic revolu-
tion” has two primary positive consequences. The first is for the customer/consumer,
who requires an increase in performance, comfort, assistance for mobility efficiency
(navigation), and safety on the one hand, while on the other hand, is seeking reduced
fuel consumption and cost. The second positive consequence is for the stakehold-
ers, carmakers, and suppliers, because software-based technology reduces marketing
time, development cost, production, and maintenance cost. Additionally, these inno-
vations have a strong impact on our society because reduced fuel consumption and
exhaust emissions improve the protection of our natural resources and the environ-
ment, while the introduction of vision systems, driver assistance, onboard diagnosis,
etc., targets a “zero death” rate, as has been stated in Australia, New Zealand, Sweden,
and the United Kingdom.
However, all these advantages are faced with an engineering challenge; there have
been an increasing number of breakdowns due to failure in electric/electronic sys-
tems. For example, Ref.6 indicates that, for 2003, 49.2\% of car breakdowns were
due to such problems in Germany. The quality of a product obviously depends on
the quality of its development, and the increasing complexity of in-vehicle embed-
ded systems raises the problem of mastering their development. The design process
is based on a strong cooperation between different players, in particular Tier 1 sup-
pliers and carmakers, which involves a specific concurrent engineering approach. For
example, in Europe or Japan, carmakers provide the specification for the subsystems
to suppliers, who, in turn, compete to find a solution for these carmakers. The chosen
suppliers are then in charge of the design and realization of these subsystems, includ-
ing the software and hardware components, and possibly the mechanical or hydraulic
parts as well. The results are furnished to the carmakers, or original equipment man-
ufacturer (OEM), who install them into the car and test them. The last step consists of
calibration activities where the control and regulation parameters are tuned to meet
the required performance of the controlled systems. This activity is closely related to
the testing activities. In the United States, this process is slightly different since the
suppliers cannot really be considered as independent from the carmakers.
Not all electronic systems have to meet the same level of dependability as the pre-
vious examples. While with a multimedia system customers require a certain quality
and performance, with a chassis control system, safety assessment is the predominant
concern. So, the design method for each subsystem depends on different techniques.
Nevertheless, they all have common distributed characteristics and they must all be
at the level of quality fixed by the market, as well as meeting the safety requirements
and the cost requirements. As there has been a significant increase in computer-
based and distributed controllers for the core critical functions of a vehicle (power
train, steering or braking systems, “X-by-wire” systems, etc.) for several years now,
a standardization process is emerging for the safety assessment and certification of
automotive-embedded systems, as has already been done for avionics and the nuclear
industry, among others. Therefore, their development and their production need to
be based on a suitable methodology, including their modeling, a priori evaluation and
validation, and testing. Moreover, due to competition between carmakers or between
suppliers to launch new products under cost, performance, reliability, and safety
constraints, the design process has to cope with a complex optimization problem.
In-vehicle embedded systems are usually classified according to domains
that correspond to different functionalities, constraints, and models. They can
be divided among “vehicle-centric” functional domains, such as power train control,
chassis control, and active or passive safety systems and “passenger centric” functional
Vehicle Functional Domains and Their Requirements
domains where multimedia/telematics, body/comfort, and human–machine interface
(HMI) can be identified.



