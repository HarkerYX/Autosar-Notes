\section{On the Scheduling of Mixed-Criticality Real-Time Task Sets}

\subsection*{Main Idea}
This paper introduces the concept of Zero slack scheduling, In zero slack scheduling the task consist of two execution modes: normal and critical. When a task is unable to meet its budget demands in normal mode, it switches to critical mode at the last instance at which it is able to meet the demand in critical mode.

\section{Schedule Table Generation for Time-Triggered Mixed Criticality Systems: Jens Theis et.al}

\subsection*{Main Idea}
Heuristic search algorithm for developing schedule tables of different criticalities.
\begin{itemize}
	\item Builds on previous work on implementing mixed criticality in Time Triggered domain.
	\item Two different schedule tables are maintained in the previous approach and priority of each job based on its criticality level.
	\item The priority ordering is inflexible resulting in total utilization which is lower than EDF.
	\item Use of mode change schedulers to accommodate two different criticality levels(Changing operational modes in the context of pre-run-time scheduling.).
	\item Two schedule tables are constructed and switching from Lo to High is possible at every point in time called \textit{Switch through property.}
	\item Each job consists of low and high mode states: \textit{J\textsubscript{lo} and J\textsubscript{hi}}. High critical definition consists of \textit{J\textsubscript{hi} }and \textit{$\delta$J\textsubscript{hi}}. 
	\item Precedence constraint introduced to prevent \textit{$\delta$J\textsubscript{hi}} from getting scheduled before \textit{J\textsubscript{lo}}.
	\item Granularity of the Scheduler is schedule table width and hence preemptible at these boundaries.
	\item Each scheduling decision for both tables represented as an edge in search tree, based on iterative deepening.
	\item Feasible schedule table design done using backtracking based tree search.
	\item Concept of \textit{leeway} is introduced as heuristic function to aid backtracking mechanism. For low criticality jobs it is the difference between the current time and its deadline. For high criticality jobs it the difference between deadline of the job in low crit mode and current time, reduced by the $\delta$J\textsubscript{hi} of all jobs that could executed between current time and deadline of $\delta$J\textsubscript{hi} of current job.
	\item During tree searching for viable schedule backtrack to previous node for a negative leeway.
			
\end{itemize}
\subsection*{Model}
\begin{itemize}
	\item \textbf{Low crit Table}: Pick Job with earliest deadline that is released --> Consider for leeway --> Backtrack if necessary.
	\item \textbf{High Crit Table}: Based on decision of low crit table, pick a Job. Pick a $\delta$J\textsubscript based on precedence constraint (J\textsubscript{lo} before respective $\delta$J\textsubscript{lo}). In case a $\delta$J missed its deadline backtrack.
\end{itemize}
\textbf{Backtracking Heuristics:}\\
Backtracking step involves swapping current node with a node higher up in tree. And is based on:
\begin{itemize}
	\item Release time should be later than release time of Job i.
	\item The leeway of possible swap slot must be greater than or equal to difference in number of slots between current slot and candidate of the swapping slot.
	\item After swapping recalculate the leeways of these tasks.
	\item If precedence constraint is changed or scheduled demand of $\delta$J is changed, start new schedule decision from swapped position.
\end{itemize}


\section{Using Dual Priority Scheduling to Improve the Resource utilization in nMRPA Microcontrollers.}
\subsection*{Main Idea}
Handling of tasks when the system enters a state that is different from normal running state for which certification of the system has been done.
\begin{itemize}
	\item Uses three different queues to separate the executing tasks, viz., run queue, interrupt queue and long task queue.
	\item tasks are picked first from active queue, then interrupt queue and from long task queue in that order.
	\item Over a heuristically determined number of periods, if same task is executing, it is moved to long task queue.
	\item Main aim of the approach to bring flexibility to the static scheduler.
\end{itemize}


\section{Time-Triggered Mixed-Critical Scheduler: Dario Socci et.al}
\subsection*{Main Idea}
A scheduling approach on single core when the exact arrival time are known a priori. This paper proposes a generalization of the Single Time Table Per Mode Scheduling proposed by Baruah et.al .
\begin{itemize}
	\item Baruah et.al proposed time triggered version of OCBP. The scheduling algorithm uses one static table per criticality mode also known as \textit{Single Time Table Per Mode}(\textbf{STTM}).
	\item 
\end{itemize}

\section{Mixed-Criticality Scheduling in Time Triggered Legacy Systems}
\subsection*{Main Idea}
A method to add handling of criticality changes to existing schedule tables for legacy TT systems.
\begin{itemize}
	\item A simple online mechanism that executes the jobs according the existing schedule table, manages a change of criticality and continues to execute the high criticality job set.
	\item Method based on slot shifting. Table and constraints are analysed and unused resources and leeways are determined. This corresponds to spare capacity.
	\item The \textit{spare capacity} is used to provide flexibility and handle firm aperiodic tasks at runtime.
	\item In the offline phase of the slot shifting complex precedence constraints are resolved.
	\item Task scheduling is considered in terms of time slots or instances. The main idea of scheduling is based on utilizing slack available in each such slot.
	Slack calculated as:\\
		SC\textsuperscript{LO}(I\textsubscript{i}) =$\arrowvert I_i \arrowvert$ - $\Sigma$ C\textsubscript{LO} + \textit{min}(sc\textsuperscript{LO}(I\textsubscript{i + 1}, 0)) \\
		SC\textsuperscript{HI}(I\textsubscript{i}) =$\arrowvert I_i \arrowvert$ - $\Sigma$ C\textsubscript{HI} + \textit{min}(sc\textsuperscript{HIs}(I\textsubscript{i + 1}, 0))
\end{itemize}
\textbf{Online Decision Process}\\
Three separate run queue are maintained: One for low criticality tasks, one for high and third one consisting of all the tasks.
During scheduling, for overrun of slack for a given slot is compensated from the immediate next slot, and recovered during idle slot.
\begin{itemize}
	\item When the low critical slack for current slot is greater than or equal to zero(s\textsubscript{lo} >= 0), schedule tasks with earliest deadline from the runqueue.
	\item If s\textsubscript{lo} < 0, Task with earliest deadline from the high crit runqueue.
	\item s\textsubscript{hi} < 0, indicates possible failure of schedule.
\end{itemize}



\section{Integrated Time- and Event-Triggered Scheduling: Overhead analysis on ARM Architecture.}
\subsection*{Main Idea}
Analyses the slot shifting approach to integrate time triggered and event triggered scheduling of Real Time Systems.
\begin{itemize}
	\item Implements a slot shifting algorithm and analyses the cost of adding flexibility to offline schedules due to addition of aperiodic tasks.
	\item Three important issues are considered: 
	\begin{itemize}
		\item While measuring runtimes, avoid distortions caused by other processes executing in the system or caused by interrupts of the hardware systems.
		\item The operating system can be a source of distortion in measurement, as processes running in parallel can trigger actions on the bus and can result in unpredictable cache behaviours.
		\item A suitable platform is required to reproduce runtime information.
	\end{itemize}
	\item MPARM hardware simulator is used for the experiments.
\end{itemize}
\subsection*{Model and Notes}
\begin{itemize}
	\item Background discusses some previous approaches in this direction, to assign aperiodic task along with periodic tasks.
	\item Discusses usage of server task to reserve bandwidth for aperiodic task. Other server algorithms are noted.Advantage being better service for aperiodic tasks at the cost of pessimistic reservation of the bandwidth.
	\item Constant bandwidth servers are similarly mentioned, which in addition to providing service to aperiodic tasks maintains temporal isolation.
	\item Slack stealing algorithm, acting as master procrastinator with periodic tasks to service aperiodic tasks, But hampered by excessing computation costs.
\end{itemize}
\subsubsection*{Slot Shifting Algorithm}
\begin{itemize}
	\item Aims for predictable flexibility on top of an offline schedule.
	\item Consists of an offline and online component.
	\item \textbf{Offline Mode}:
	\begin{itemize}
		\item Create distributed schedule table with precedence.
		\item Determine earliest start time and latest finishing time for all tasks.
		\item On each node, deadlines of all tasks are sorted and offline schedule is divided into disjoint intervals(In AUTOSAR world: Schedule Table.), with each interval consisting of tasks with same deadline.
		\item Slack in system defined as:\\
		sc(\textit{I\textsubscript{j}}) = |\textit{I\textsubscript{j}}| - $\Sigma$ \textit{WCET(T\textsubscript{i})} + \textit{min(sc(I\textsubscript{j+1}), 0)}, Where \textit{I\textsubscript{j}} is the effective width of time interval.
		\item So slack is interval width decreased by sum of WCET of tasks belonging to given interval and amount of slot/slack lent to next time interval.
	\end{itemize}
\item \textbf{Online Mode}:
	\begin{itemize}
		\item For each new slot(strict, periodically repeating, time-interval. Roughly equivalent to time-tick cost + a non-preemptible width.), scheduler invoked on each node independently. Each slot having scheduling cost(t\textsubscript{s}) and execution budget(t\textsubscript{e}).
		\item Each schedule point, check for aperiodic task arrival, \textbf{check for total slack available till deadline of aperiodic task, if it meets the task demand run the aperiodic task else reject.}
		\item Similarly for an idle instance, run aperiodic task, decrease from current time interval.
	\end{itemize}
\end{itemize}
\subsubsection*{Implementation}
\begin{itemize}
	\item \textit{task\_list} and \textit{interval\_list}: interval list maintains grouped tasks and trigger events.
	\item Timer ISR to run the decision algorithm on aperiodic tasks.
	\item Aperiodic tasks arrived handled in earliest deadline first.
	\item Accommodating aperiodic task done either by running in background or recreating the tables till the aperiodic task deadline.(Done only if task passes acceptance test.)	
\end{itemize}
\subsubsection*{Experiment}
\begin{itemize}
	\item Simulator based test bed.
	\item logging in shared memory and dump to interface to reduce overhead.
	\item Experiment 1: Overhead of slot shifting online algorithm.
	\item Experiment 2: Overhead of online splitting of schedule table.
	\item Experiment 3: Optimizing splitting overhead by aligning aperiodic task to existing time interval boundary.
	\item Experiment 4: Scalability on multicore.
	\item Aggregated histogram plot on logging overhead, ISR overhead, schedule list updation overhead, acceptance test overhead, polling for aperiodic task overhead.
	\item Memory overhead analysis.
\end{itemize}

\section{RTOS support for mixed time-triggered and event-triggered task sets}
\subsection*{Main Idea}
Extend an off the shelf RTOS(In this case $\mu$C/OS-II) with time triggered implementation and slot shifting to support time-triggered and event triggered tasks.\\
Tasks executed based on earliest deadline first and aperiodic tasks accommodated  if it passes a predefined acceptance test(As is the case in above paper using slack).
\begin{itemize}
	\item Implements dynamic slot shifting.
	\item Additionally implements an enhanced mechanism for slack reclamation from periodic tasks.(Based on the fact that periodic task may finish before there specified WCET, thus proving more slack than computed based on an offline algorithm.) 
\end{itemize} 
\subsubsection*{Some key differentiation of slot shifting from other approaches}
\begin{itemize}
	\item Similar to the reservation based approach, distinguishing characteristics being:
	\begin{itemize}
		\item Lack of temporal isolation between tasks.
		\item Tasks can execute over multiple budget(or Time Intervals, or Schedule Tables as is case with AUTOSAR).
	\end{itemize}
	\item Compared to hierarchical scheduling, provisioned spare capacity neither follows a periodic service pattern nor provides a constant service rate. Disjunct timing intervals created rather than regular intervals as in case of hierarchical approach.
\end{itemize}
\subsection*{Model and Notes}
\begin{itemize}
	\item New structure introduced similar to Task Control Block(TCB) - Interval Control Block(ICB), to store offline specified task intervals with tasks, count, start time, end time, spare capacity etc.
	\item  Resource management done at tick ISR.
	\item TCB extended to keep track of execution budget assigned/remaining and ICB to which the task belongs to.
	\item Experiment section involves comparison of overhead in slot shifting with respect to Fixed Priority Scheduling and Earliest Deadline First Scheduling.
\end{itemize}
