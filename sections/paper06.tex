\section{Scheduling Algorithms and OS Support for RTS.}
\subsection{Key Idea}
An overview of the scheduling support in real time systems and subsequent use cases in commercial of the shelf RTOS.
Three main type of scheduling paradigms are discussed:
\begin{itemize}
	\item Static table driven approaches.
	\item Static priority driven approaches.
	\item Dynamic planning based approaches.
	\item Dynamic best effort approaches.
\end{itemize}
Discussed are table driven scheduling, priority based preemptive scheduling, cyclic scheduling.
Cyclic scheduling resembles the schedule table approach used in autosar based systems. Here tasks are assigned one of a set of harmonic periods. 
If purely priority driven approach is used, by using task deadlines as priorities, and without any planning, task would be preempted at any time.
So unless the deadline is reached or task completes, whichever comes first, It is not possible to know if the task constraints are met.
So worst case performance analysis of such a task system is necessary.\\
When tasks are acted upon by parameters other than priority like resource requirements heuristics based approach on branch and bound are necessary.
\subsection{Ideas}
Priority categorization of I/O bound and CPU bound tasks in the context on non-realtime systems.
\begin{itemize}
	\item In addition to being intuitive, static priority assignment forgoes re-computation of the priority at runtime.
	\item Similar to EDF, another dynamic approach to scheduling is based on laxity, or least laxity first scheduling.
	\item Although feasibility checking on schedulability is made easier by preemption, most schedulability analysis ignores the dispatch cose associated with scheduling, which can be significant depending upon the task characteristics.
	\item Dynamic planning based algorithm: OCBP can be considered a variant of this approach. In this approach a task is guaranteed by constructing a plan for the all the tasks that are available at its possession at the given time.
	\item Different mechanisms to dynamic planning algorithms exist: Focused addressing, bidding algorithm, flexible algorithm or a combination of bidding/focused algorithm.
	\item Dynamic best effort algorithm: Used in most of the multicore real time systems. Example being earliest deadline first and the least laxity first algorithms.
\end{itemize}
\subsection{Comclusion}
Paper is representation of work in the area of scheduling and operating system support for the same.

