%A Heuristic algorithm for mapping autosar runnables to tasks.
%\textbf{https://hal.archives-ouvertes.fr/hal-01341785/document}
\section{A Novel heuristic Algorithm For Mapping AUTOSAR Runnables To Tasks.}

\subsection*{Key Idea}
Runnables represents the internal behavior of SwC in AUTOSAR and are the smallest pieces of the code to be scheduled.
All runnables are triggered in response to an event such as timing event, data receiving or operations invoking the runnables.
Key question is deciding upon how many tasks are required to represent a set of runnables, how to define priority and how to define activation offset and execution order of the tasks.\\
Prominent methods of grouping the runnables to tasks are:
\begin{itemize}
	\item \textbf{{Periodic Solution}}: Runnables with same period are assigned to one task. WCET of the task is sum of the WCET of all the runnables mapped to it.
	\item \textbf{Multiple Periodic Solution}: In this approach period of the task is the shortest period among different runnables mapped to it.
	\item \textbf{Arbitrary Periodic Solution}: In this approach runnables with different approach are mapped to same task using their activation offset. period of the task is GCD of all the runnables mapped to it.
\end{itemize}
For APS and MPS system schedulability may be guaranteed using Rate Monotonic Period Assignment(RMPA) or Deadline Monotonic Period Assignment(DMPA).
Currently multiple approach to mapping exists, including use of MILP, Simulated Annealing and Genetic algorithm based approach.

\subsection*{Width and Scope}
Using Audsley's approach to sort out the tasks among different schedule frames with constraint of width of the frame and the deadline of each of the runtimes.

\subsection*{Conclusion}
Neat way, but practicality is limited, as usual runtimes of tasks are grouped to common schedule tables as there functionalities are important.