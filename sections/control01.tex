\section{Power-Steering Control Architecture for Automatic Driving}
A two layered control architecture to automatically moving the steering wheel is presented.\\
\subsection*{Overview}
\begin{itemize}
	\item Two layer control architecture, first layer is designed to calculate the position of the steering wheel at any time based on fuzzy logic.
	\item The second layer is a classic control layer that moves the steering wheel to the position predicted by the first layer, monitored by Real-Time Kinematic Differential Global Positioning System(RTK-DGPS).
	\item Comparison is done for the performance of the implementation to human driver. 
\end{itemize}

\subsection*{Key points}
\begin{itemize}
	\item There are two ways to implement automatic steering control, Imitating a driver or using dynamic models of car and control methods based on linear control theory.
	\item  the main contribution
	of this work is the combined use of a PID and a fuzzy controller
	in a cascade-control scheme, and its application to regulate the
	steering wheel of a mass-produced vehicle.
	\item The sensor input comes from GPS receiver at a frequency of 10Hz.
	\item Each measured value is compared with reference GPS trajectory. Two input variables are gathered from this comparison: the lateral and angular errors.
	\item A Fuzzy control ingests these two inputs to generate target steering turning command.
	\item Fuzzy control performs three main functionalities: fuzzification, inference, defuzzification. 
	\item Two different driving conditions are handled. Straight road and curved road.
	\item Straight road steering angles are considered short and fast reactive while curve is considered higher steering angles and slower reaction.
	\item Steering angle is limited to 2.5\% on straight road. While in curve mode the angle is not limited.
	\item Center-of-area method is used for defuzzification.
	\item Each crisp value, which is left or right steering output is multiplied by square of weight and averaged over sum of weights.
	\item The weights are calculated using Mamdani's rule inference method.
	\item The trajectory control defined above is fed to a PID controller that is tuned on the basis of Zeigler-Nichols method and further tuned experimentally, proportional, integral and derivative gains were adjusted to yield an overdamped closed-loop response.
	\item Proportional, integral and derivative gains were set to 20, 50 and 1000 respectively.	
\end{itemize}

\subsection*{Pseudo codes}
\begin{algorithm}
	\caption{First layer Fuzzy Controller}\label{fuzzy}
	\begin{algorithmic}[1]
		\Procedure{Fuzzy\_Rule()}{}
		\If{\textit{Angular\_Error == Left} || \textit{Lateral\_Error == Left}} \Return Right
		\EndIf
		\If{\textit{Angular\_Error == Right} ||\textit{Lateral\_Error == Right}} \Return Left
		\EndIf
		\EndProcedure
	\end{algorithmic}
\end{algorithm}

\subsection*{reference}
Paper Link: